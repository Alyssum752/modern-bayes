\documentclass[mathserif]{beamer}

\setbeamertemplate{frametitle}[default][center]%Centers the frame title.
\setbeamertemplate{navigation symbols}{}%Removes navigation symbols.
\setbeamertemplate{footline}{\raisebox{5pt}{\makebox[\paperwidth]{\hfill\makebox[10pt]{\scriptsize\insertframenumber}}}}
\setbeamertemplate{caption}[numbered]

\usepackage{amssymb,amsfonts,amsmath,latexsym,amsthm}
%\usepackage[usenames,dvipsnames]{color}
%\usepackage[]{graphicx}
%\usepackage[space]{grffile}
\usepackage{mathrsfs}   % fancy math font
% \usepackage[font=small,skip=0pt]{caption}
\usepackage[skip=0pt]{caption}
\usepackage{subcaption}
\usepackage{verbatim}
\usepackage{url}
\usepackage{bm}
\usepackage{dsfont}
\usepackage{extarrows}
\usepackage{multirow, bm}
%\newcommand{\tth}   {\mbox{$\theta$}}
\newcommand{\thh}   {\mbox{$\theta$}}
\newcommand{\su}   {\mbox{$\sigma^2$}}
\newcommand{\so}   {\mbox{$\sigma_0^2$}}
\newcommand{\ko}   {\mbox{$\kappa_0$}}
\newcommand{\no}   {\mbox{$\nu_0$}}
\newcommand{\mo}   {\mbox{$\mu_0$}}
\newcommand{\ti}   {\mbox{$\tilde{x}$}}
\newcommand{\la}   {\mbox{$\lambda$}}
\newcommand{\bx}   {\mbox{$\bm{x}$}}
\newcommand{\bZ}   {\mbox{$\bm{Z}$}}
\newcommand{\bX}   {\mbox{$\bm{X}$}}
\newcommand{\bY}   {\mbox{$\bm{Y}$}}
\newcommand{\bA}   {\mbox{$\bm{A}$}}
\newcommand{\ba}   {\mbox{$\bm{a}$}}
\newcommand{\bb}   {\mbox{$\bm{b}$}}
\newcommand{\bt}   {\mbox{$\bm{t}$}}
\newcommand{\bz}   {\mbox{$\bm{z}$}}
\newcommand{\bw}   {\mbox{$\bm{w}$}}
\newcommand{\bbeta}   {\mbox{$\bm{\beta}$}}

\newcommand{\be}   {\mbox{$\bm{e}$}}
\newcommand{\bu}   {\mbox{$\bm{u}$}}
\newcommand{\bv}   {\mbox{$\bm{v}$}}
\newcommand{\sig}   {\mbox{$\Sigma$}}
\newcommand{\sigx}   {\mbox{$\Sigma_{XX}$}}
\newcommand{\sigxy}   {\mbox{$\Sigma_{XY}$}}
\newcommand{\tr}   {\mbox{$\text{tr}$}}
\newcommand{\ddet}   {\mbox{$\text{det}$}}
\newcommand\independent{\protect\mathpalette{\protect\independenT}{\perp}}
\def\independenT#1#2{\mathrel{\rlap{$#1#2$}\mkern2mu{#1#2}}}

\newcommand{\Expect}[1]{\ensuremath{\mathbf{E}\left[ #1 \right]}}
%\newcommand{\Var}[1]{\ensuremath{\mathrm{Var}\left[ #1 \right]}}
%\newcommand{\Cov}[1]{\ensuremath{\mathrm{Cov}\left[ #1 \right]}}
\newcommand{\MSE}{\ensuremath{\mathrm{MSE}}}
\newcommand{\RSS}{\ensuremath{\mathrm{RSS}}}
\newcommand{\Prob}[1]{\ensuremath{\mathrm{Pr}\left( #1 \right)}}
\newcommand{\ProbEst}[1]{\ensuremath{\widehat{\mathrm{Pr}}\left( #1 \right)}}
\DeclareMathOperator*{\argmin}{argmin} % thanks, wikipedia!
\DeclareMathOperator*{\argmax}{argmax} % thanks, wikipedia!
\DeclareMathOperator*{\sgn}{sgn} % thanks, wikipedia!

\newcommand{\lam}{\lambda}
\newcommand{\bmu}{\bm{\mu}}
%\newcommand{\bx}{\ensuremath{\mathbf{X}}}
\newcommand{\X}{\ensuremath{\mathbf{X}}}
\newcommand{\w}{\ensuremath{\mathbf{w}}}
\newcommand{\h}{\ensuremath{\mathbf{h}}}
\newcommand{\V}{\ensuremath{\mathbf{V}}}
%\newcommand{\tr}{\operatorname{tr}}

%\newcommand{\bx}{\ensuremath{\mathbf{X}}}
%\newcommand{\X}{\ensuremath{\mathbf{x}}}
%\newcommand{\w}{\ensuremath{\mathbf{w}}}
%\newcommand{\h}{\ensuremath{\mathbf{h}}}
%\newcommand{\V}{\ensuremath{\mathbf{v}}}
%\newcommand{\Cov}{\text{Cov}}
%\newcommand{\Var}{\text{Var}}

\DeclareMathOperator{\var}{Var}
\DeclareMathOperator{\cov}{Cov}
\newcommand{\Var}[1]{\ensuremath{\mathrm{Var}\left[ #1 \right]}}
\newcommand{\Cov}[1]{\ensuremath{\mathrm{Cov}\left[ #1 \right]}}


\newcommand{\indep}{\rotatebox{90}{\ensuremath{\models}}}
\newcommand{\notindep}{\not\hspace{-.05in}\indep}







\usepackage{float,bm}
\floatstyle{boxed}
\newfloat{code}{tp}{code}
\floatname{code}{Code Example}
%\newcommand{\tth}   {\mbox{$\theta$}}
\newcommand{\thh}   {\mbox{$\theta$}}
\newcommand{\su}   {\mbox{$\sigma^2$}}
\newcommand{\so}   {\mbox{$\sigma_0^2$}}
\newcommand{\ko}   {\mbox{$\kappa_0$}}
\newcommand{\no}   {\mbox{$\nu_0$}}
\newcommand{\mo}   {\mbox{$\mu_0$}}
\newcommand{\ti}   {\mbox{$\tilde{x}$}}
\newcommand{\la}   {\mbox{$\lambda$}}
\newcommand{\bx}   {\mbox{$\bm{x}$}}
\newcommand{\bZ}   {\mbox{$\bm{Z}$}}
\newcommand{\bX}   {\mbox{$\bm{X}$}}
\newcommand{\bY}   {\mbox{$\bm{Y}$}}
\newcommand{\bA}   {\mbox{$\bm{A}$}}
\newcommand{\ba}   {\mbox{$\bm{a}$}}
\newcommand{\bb}   {\mbox{$\bm{b}$}}
\newcommand{\bt}   {\mbox{$\bm{t}$}}
\newcommand{\bz}   {\mbox{$\bm{z}$}}
\newcommand{\bw}   {\mbox{$\bm{w}$}}
\newcommand{\bbeta}   {\mbox{$\bm{\beta}$}}

\newcommand{\be}   {\mbox{$\bm{e}$}}
\newcommand{\bu}   {\mbox{$\bm{u}$}}
\newcommand{\bv}   {\mbox{$\bm{v}$}}
\newcommand{\sig}   {\mbox{$\Sigma$}}
\newcommand{\sigx}   {\mbox{$\Sigma_{XX}$}}
\newcommand{\sigxy}   {\mbox{$\Sigma_{XY}$}}
\newcommand{\tr}   {\mbox{$\text{tr}$}}
\newcommand{\ddet}   {\mbox{$\text{det}$}}
\newcommand\independent{\protect\mathpalette{\protect\independenT}{\perp}}
\def\independenT#1#2{\mathrel{\rlap{$#1#2$}\mkern2mu{#1#2}}}

\newcommand{\Expect}[1]{\ensuremath{\mathbf{E}\left[ #1 \right]}}
%\newcommand{\Var}[1]{\ensuremath{\mathrm{Var}\left[ #1 \right]}}
%\newcommand{\Cov}[1]{\ensuremath{\mathrm{Cov}\left[ #1 \right]}}
\newcommand{\MSE}{\ensuremath{\mathrm{MSE}}}
\newcommand{\RSS}{\ensuremath{\mathrm{RSS}}}
\newcommand{\Prob}[1]{\ensuremath{\mathrm{Pr}\left( #1 \right)}}
\newcommand{\ProbEst}[1]{\ensuremath{\widehat{\mathrm{Pr}}\left( #1 \right)}}
\DeclareMathOperator*{\argmin}{argmin} % thanks, wikipedia!
\DeclareMathOperator*{\argmax}{argmax} % thanks, wikipedia!
\DeclareMathOperator*{\sgn}{sgn} % thanks, wikipedia!

\newcommand{\lam}{\lambda}
\newcommand{\bmu}{\bm{\mu}}
%\newcommand{\bx}{\ensuremath{\mathbf{X}}}
\newcommand{\X}{\ensuremath{\mathbf{X}}}
\newcommand{\w}{\ensuremath{\mathbf{w}}}
\newcommand{\h}{\ensuremath{\mathbf{h}}}
\newcommand{\V}{\ensuremath{\mathbf{V}}}
%\newcommand{\tr}{\operatorname{tr}}

%\newcommand{\bx}{\ensuremath{\mathbf{X}}}
%\newcommand{\X}{\ensuremath{\mathbf{x}}}
%\newcommand{\w}{\ensuremath{\mathbf{w}}}
%\newcommand{\h}{\ensuremath{\mathbf{h}}}
%\newcommand{\V}{\ensuremath{\mathbf{v}}}
%\newcommand{\Cov}{\text{Cov}}
%\newcommand{\Var}{\text{Var}}

\DeclareMathOperator{\var}{Var}
\DeclareMathOperator{\cov}{Cov}
\newcommand{\Var}[1]{\ensuremath{\mathrm{Var}\left[ #1 \right]}}
\newcommand{\Cov}[1]{\ensuremath{\mathrm{Cov}\left[ #1 \right]}}


\newcommand{\indep}{\rotatebox{90}{\ensuremath{\models}}}
\newcommand{\notindep}{\not\hspace{-.05in}\indep}






%\usepackage{fontspec}
%\setmainfont{Tahoma}

%\newcommand{\lam}{\lambda}
\newcommand{\bmu}{\bm{\mu}}
\newcommand{\bX}   {\bm{X}}
\newcommand{\sig}   {\Sigma}
\newcommand{\bx}{\ensuremath{\mathbf{X}}}
%\newcommand{\X}{\ensuremath{\mathbf{x}}}
%\newcommand{\w}{\ensuremath{\mathbf{w}}}
%\newcommand{\h}{\ensuremath{\mathbf{h}}}
%\newcommand{\V}{\ensuremath{\mathbf{v}}}
%\newcommand{\cov}{\text{Cov}}
\newcommand{\var}{\text{Var}}

%\DeclareMathOperator{\var}{Var}
%\DeclareMathOperator{\cov}{Cov}

%\newcommand{\indep}{\rotatebox{90}{\ensuremath{\models}}}
%\newcommand{\notindep}{\not\hspace{-.05in}\indep}

\usepackage{graphicx} %The mode "LaTeX => PDF" allows the following formats: .jpg  .png  .pdf  .mps
\graphicspath{{./PresentationPictures/}} %Where the figures folder is located
\usepackage{listings}
\usepackage{media9}
\usepackage{movie15}
\addmediapath{./Movies/}

\newcommand{\beginbackup}{
   \newcounter{framenumbervorappendix}
   \setcounter{framenumbervorappendix}{\value{framenumber}}
}
\newcommand{\backupend}{
   \addtocounter{framenumbervorappendix}{-\value{framenumber}}
   \addtocounter{framenumber}{\value{framenumbervorappendix}} 
}


%\usepackage{algorithm2e}
\usepackage[ruled,lined]{algorithm2e}
\def\algorithmautorefname{Algorithm}
\SetKwIF{If}{ElseIf}{Else}{if}{then}{else if}{else}{endif}
%\usepackage{times}
%\usepackage[tbtags]{amsmath}
%\usepackage{amssymb}
\usepackage{amsfonts}
%\usepackage{slfortheorems}
\usepackage{epsfig}
\usepackage{graphicx}
%\usepackage[small]{caption}
%\usepackage[square]{natbib}
%\newcommand{\newblock}{}
%\bibpunct{(}{)}{;}{a}{}{,}
%\bibliographystyle{ims}
%\usepackage[letterpaper]{geometry}
%\usepackage{color}
%\setlength{\parindent}{0pt}

\usepackage{natbib}
\bibpunct{(}{)}{;}{a}{}{,}
%\usepackage{hyperref}

\DeclareMathOperator*{\Exp}{Exp}
\DeclareMathOperator*{\TExp}{TExp}
\DeclareMathOperator*{\Bernoulli}{Bernoulli}
\DeclareMathOperator*{\Beta}{Beta}
\DeclareMathOperator*{\Ga}{Gamma}
\DeclareMathOperator*{\TGamma}{TGamma}
\DeclareMathOperator*{\Poisson}{Poisson}
\DeclareMathOperator*{\Binomial}{Binomial}
\DeclareMathOperator*{\NormalGamma}{NormalGamma}
\DeclareMathOperator*{\InvGamma}{InvGamma}
\DeclareMathOperator*{\Cauchy}{Cauchy}
\DeclareMathOperator*{\Uniform}{Uniform}
\DeclareMathOperator*{\Gumbel}{Gumbel}
\DeclareMathOperator*{\Pareto}{Pareto}
\DeclareMathOperator*{\Mono}{Mono}
\DeclareMathOperator*{\Geometric}{Geometric}
\DeclareMathOperator*{\Wishart}{Wishart}

\newcommand{\N}{\mathcal{N}}

\newcommand{\R}{\mathbb{R}}
\newcommand{\Z}{\mathbb{Z}}
\newcommand{\E}{\mathbb{E}}
\renewcommand{\Pr}{\mathbb{P}}
\newcommand{\I}{\mathds{1}}
\newcommand{\V}{\mathbb{V}}

% Math operators
\DeclareMathOperator*{\diag}{diag}
\DeclareMathOperator*{\median}{median}
\DeclareMathOperator*{\Vol}{Vol}

% Miscellaneous commands
\newcommand{\iid}{\stackrel{\mathrm{iid}}{\sim}}
\newcommand{\matrixsmall}[1]{\bigl(\begin{smallmatrix}#1\end{smallmatrix} \bigr)}

\newcommand{\items}[1]{\begin{itemize} #1 \end{itemize}}

\newcommand{\todo}[1]{\emph{\textcolor{red}{(#1)}}}

\newcommand{\branch}[4]{
\left\{
	\begin{array}{ll}
		#1  & \mbox{if } #2 \\
		#3 & \mbox{if } #4
	\end{array}
\right.
}

% approximately proportional to
\def\app#1#2{%
  \mathrel{%
    \setbox0=\hbox{$#1\sim$}%
    \setbox2=\hbox{%
      \rlap{\hbox{$#1\propto$}}%
      \lower1.3\ht0\box0%
    }%
    \raise0.25\ht2\box2%
  }%
}
\def\approxprop{\mathpalette\app\relax}

\newcommand{\btheta}{{\bm\theta}}
\newcommand{\bbtheta}{{\pmb{\bm\theta}}}

%\usepackage{zref-savepos}
%
%\newcounter{restofframe}
%\newsavebox{\restofframebox}
%\newlength{\mylowermargin}
%\setlength{\mylowermargin}{2pt}
%
%\newenvironment{restofframe}{%
%    \par%\centering
%    \stepcounter{restofframe}%
%    \zsavepos{restofframe-\arabic{restofframe}-begin}%
%    \begin{lrbox}{\restofframebox}%
%}{%
%    \end{lrbox}%
%    \setkeys{Gin}{keepaspectratio}%
%    \raisebox{\dimexpr-\height+\ht\strutbox\relax}[0pt][0pt]{%
%    \resizebox*{!}{\dimexpr\zposy{restofframe-\arabic{restofframe}-begin}sp-\zposy{restofframe-\arabic{restofframe}-end}sp-\mylowermargin\relax}%
%        {\usebox{\restofframebox}}%
%    }%
%    \vskip0pt plus 1filll\relax
%    \mbox{\zsavepos{restofframe-\arabic{restofframe}-end}}%
%    \par
%}


\usepackage{tikz}
\usetikzlibrary{arrows}

%\usepackage[usenames,dvipsnames]{xcolor}
\usepackage{tkz-berge}
\usetikzlibrary{fit,shapes}

\usepackage{calc}
%%
%% The tikz package is used for doing the actual drawing.
%\usepackage{tikz}
%%
%% In order to be able to put arrowheads in the middle of directed edges, we need an extra library.
\usetikzlibrary{decorations.markings}
%%
%% The next line says how the "vertex" style of nodes should look: drawn as small circles.
\tikzstyle{vertex}=[circle, draw, inner sep=0pt, minimum size=6pt]
%%
%% Next, we make a \vertex command as a shorthand in place of \node[vertex} to get that style.
\newcommand{\vertex}{\node[vertex]}
%%
%% Finally, we declare a "counter", which is what LaTeX calls an integer variable, for use in
%% the calculations of angles for evenly spacing vertices in circular arrangements.
\newcounter{Angle}

\newtheoremstyle{example}
{\topsep} % space above
{\topsep} % space below
{} % body font
{} % indent
{\bf} % head font
{:} % punctuation between head and body
{0.5em} % space after head
{} % manually specify head
%{\thmname{#1}\thmnumber{ #2}\thmnote{:#3}} % manually specify head

\theoremstyle{example}
\newtheorem{ex}{Example}[section]

\newtheoremstyle{definition}
{\topsep} % space above
{\topsep} % space below
{} % body font
{} % indent
{\sc} % head font
{:} % punctuation between head and body
{0.5em} % space after head
{} % manually specify head
%{\thmname{#1}\thmnumber{ #2}\thmnote{:#3}} % manually specify head

\theoremstyle{definition}
\newtheorem{defn}{Definition}[section]

\theoremstyle{rem}
\newtheorem{rem}{Remark}[section]

\newtheoremstyle{theorem}
{\topsep} % space above
{\topsep} % space below
{} % body font
{} % indent
{\sc} % head font
{:} % punctuation between head and body
{0.5em} % space after head
{} % manually specify head
%{\thmname{#1}\thmnumber{ #2}\thmnote{:#3}} % manually specify head

\theoremstyle{theorm}
\newtheorem{thm}{Theorem}[section]



%%%to add in new counter for slides in beamer

%\setbeamertemplate{footline}{
%  \leavevmode%
%  \hbox{%
%  \begin{beamercolorbox}[wd=.333333\paperwidth,ht=2.25ex,dp=1ex,center]{author in head/foot}%
%    \usebeamerfont{author in head/foot}\insertshortauthor~~(\insertshortinstitute)
%  \end{beamercolorbox}%
%  \begin{beamercolorbox}[wd=.333333\paperwidth,ht=2.25ex,dp=1ex,center]{title in head/foot}%
%    \usebeamerfont{title in head/foot}\insertshorttitle
%  \end{beamercolorbox}%
%  \begin{beamercolorbox}[wd=.333333\paperwidth,ht=2.25ex,dp=1ex,right]{date in head/foot}%
%    \usebeamerfont{date in head/foot}\insertshortdate{}\hspace*{2em}
%    \insertframenumber{} \hspace*{2ex} % hier hat's sich ge�ndert
%  \end{beamercolorbox}}%
%  \vskip0pt%
%}



%%%%%

\newcommand*\oldmacro{}
\let\oldmacro\insertshortauthor
\renewcommand*\insertshortauthor{
  \leftskip=.3cm
\insertframenumber\,/\,\inserttotalframenumber\hfill\oldmacro}




%\excludecomment{notbeamer}
%\includecomment{beamer}



\title{The Multivariate Distributions: Normal and inverse Wishart}
\author{Rebecca C. Steorts \\ Bayesian Methods and Modern Statistics: STA 360/601}
\date{Module 10}

\begin{document}

\maketitle

\frame{

\begin{itemize}
\item Moving from univariate to multivariate distributions. 
\item The multivariate normal (MVN) distribution.
\item Conjugate for the MVN distribution.  
\item The inverse Wishart distribution. 
\item Conjugate for the MVN distribution (but on the covariance matrix). 
\item Combining the MVN with inverse Wishart. 
\end{itemize}

}




\section{Distribution of MVN}

\frame{
We assume that the population mean is $\bmu = E(\bX)$ and $\Sigma = \var(\bX) = E[(\bX - \bmu)(\bX - \bmu)^T],$ 
where 
\[  \bmu= \left( \begin{array}{c}
\mu_1\\
\mu_2\\
\vdots\\
\mu_p
\end{array} \right) \]
and 

$$\sig = 
\left( \begin{array}{cccc}
\sigma_1^2 & \sigma_{12} & \ldots&  \sigma_{1p}\\
\sigma_{21} & \sigma_2^2 & \ldots& \sigma_{2p}\\
\vdots & \vdots & \ddots & \vdots \\
\sigma_{p1} & \sigma_{p2} &\ldots& \sigma_p^2
\end{array} \right).
$$


}

\frame{
\frametitle{Notation}

Determinant and Trace of Matrices

}

\frame{
Suppose matrix $A$  is invertible. The 
$$\det(A) = \sum_{i=1}^{j=n} a_{ij} A_{ij}.$$
\vskip 1em 

I recommend using the det() commend in R. 
\vskip 1em 

Suppose now we have a square matrix $H_{p \times p}.$

$$\text{trace}(H) = \sum_i h_{ii},$$ where $h_{ii}$ are the diagonal elements of $H.$








}

%\frame{
%
%Define $\sigma_j^2 = \var(X_j)$ and $\sigma_{jk} = \cov(X_j, X_k).$
%
%Assume
%$\bX \sim (\bmu, \Sigma).$
%
%\begin{rem}
%Let $$\bX \sim (\bmu, \Sigma)$$ be a $p$-variate random vector. Let $A_{p\times k}$ and $B_{p\times \ell}$ be matrices. Then
%\begin{enumerate}
%\item The mean and covariance of $A^T\bX$ are
%$$A^T\bX \sim (A^T\bmu, A^T \Sigma A). $$
%\item The random vectors $A^T\bX$ and $B^T\bX$ are uncorrelated if and only if $A^T \Sigma B = 0_{k \times \ell}.$
%\end{enumerate}
%\end{rem}
%
%
%}

\frame{
\begin{itemize}
\item MVN is generalization of univariate normal.
\item For the MVN, we write $\bX \sim
\mathcal{MVN}(\bmu,\Sigma)$. 
\item The $(i,j)^{\text{th}}$
component of $\Sigma$ is the covariance between $X_i$ and~$X_j$ (so
the diagonal of $\Sigma$ gives the component variances).
\end{itemize}

}

\frame{
Just as the probability density of a scalar normal is
\begin{equation}
p(x) = {\left(2\pi\sigma^2\right)}^{-1/2}\exp{\left\{ -\frac{1}{2} \frac{(x-\mu)^2}{\sigma^2}\right\}},
\end{equation}
the probability density of the multivariate normal is
\begin{equation}
p(\vec{x}) = {\left(2\pi\right)}^{-p/2}(\det{\Sigma})^{-1/2} \exp{\left\{-\frac{1}{2} (\bx-\bmu)^T\Sigma^{-1} (\bx - \bmu)\right\}}.
\end{equation}
\textcolor{blue}{Univariate normal is special case of the multivariate normal with a one-dimensional mean ``vector'' and a one-by-one variance ``matrix.''}

}

\frame{

Please review the first section for Chapter 7 if you're uncomfortable with the density of the standard normal or MVN.


}

\frame{
\frametitle{Conjugate to MVN}
Suppose that $$X_1 \ldots X_n \stackrel{iid}{\sim} MVN(\theta, \Sigma). $$
Let $$\pi(\btheta) \sim MVN(\bmu, \Omega). $$

What is the full conditional distribution of $\btheta \mid \bX, \Sigma$?
}

\frame{
\frametitle{Prior}
\begin{align}
\pi(\btheta) &= {\left(2\pi\right)}^{-p/2}\det{\Omega}^{-1/2} \exp{\left\{-\frac{1}{2} (\btheta-\bmu)^T\Omega^{-1} (\btheta - \bmu)\right\}} \\
& \propto \exp{\left\{-\frac{1}{2} (\btheta-\bmu)^T\Omega^{-1} (\btheta - \bmu)\right\}} \\
& \propto \exp-\frac{1}{2} {\left \{\btheta^T\Omega^{-1} \btheta - 2 \btheta^T \Omega^{-1} \mu + \mu^T \Omega^{-1} \mu \right \}} \\
& \propto \exp-\frac{1}{2} {\left \{\btheta^T\Omega^{-1} \btheta - 2 \btheta^T \Omega^{-1} \mu  \right \}}\\
&= \exp-\frac{1}{2} {\left \{\btheta^TA_o \btheta - 2 \btheta^T  b_o  \right \}}
\end{align}
$\pi(\btheta) \sim MVN(\bmu, \Omega)$ implies that $A_o = \Omega^{-1}$ and $b_o = \Omega^{-1} \mu.$
}

\frame{
\frametitle{Likelihood}
\begin{align}
p(\bx \mid \btheta, \Sigma) &= \prod_{i=1}^n
{\left(2\pi\right)}^{-p/2}\det{\Sigma}^{-n/2} \exp{\left\{-\frac{1}{2} (x_i-\btheta)^T\Sigma^{-1} (x_i - \btheta)\right\}}\\
\propto 
& \exp-\frac{1}{2} {\left \{ \sum_i x_i^T \Sigma^{-1} x_i -2 \sum_i \btheta^T \Sigma^{-1} x_i + 
\sum_i \btheta^T\Sigma^{-1} \btheta 
 \right \}}\\
 & \exp-\frac{1}{2} {\left \{  -2 \btheta^T \Sigma^{-1} n\bar{x} + 
n \btheta^T\Sigma^{-1} \btheta 
 \right \}}\\
  & \exp-\frac{1}{2} {\left \{  -2 \btheta^T b_1+ 
\btheta^T A_1 \btheta \right \}},
\end{align}
where $$b_1= \Sigma^{-1} n\bar{x}, \quad A_1 = n\Sigma^{-1}$$ and 
$$\bar{x} := (\frac{1}{n}\sum_i x_{i1} ,\ldots, \frac{1}{n} x_{ip})^T.$$



}

\frame{
\frametitle{Full conditional}
\begin{align}
p(\btheta \mid \bx, \Sigma) &\propto
p(\bx \mid \btheta, \Sigma) \times p(\btheta) \\
&\propto 
\exp-\frac{1}{2} {\left \{  -2 \btheta^T b_1+ 
\btheta^T A_1 \btheta \right \}} \\
&\times
\exp-\frac{1}{2} {\left \{\btheta^TA_o \btheta - 2 \btheta^T b_o  \right \}}\\
%%%
&\propto \exp\{\btheta^T b_1 - \frac{1}{2}\btheta^T A_1 \btheta- \frac{1}{2}\btheta^TA_o  \btheta
+ \btheta^T b_o\}\\
& \propto\exp\{
\btheta^T( b_o + b_1) -\frac{1}{2}\theta^T(A_o + A_1) \theta
\}
\end{align}
Then $$A_n = A_o + A_1 = \Omega^{-1}+n\Sigma^{-1}$$ and $$b_n = b_o + b_1 = \Omega^{-1}\mu + \Sigma^{-1} n\bar{x}$$
$$\btheta \mid \bx, \Sigma \sim MVN(A_n^{-1}b_n, A_n^{-1}) = MVN(\mu_n, \Sigma_n)$$
}

\frame{
\frametitle{Interpretations?}
$$\btheta \mid \bx, \Sigma \sim MVN(A_n^{-1}b_n, A_n^{-1}) = MVN(\mu_n, \Sigma_n)$$
\begin{align}
\mu_n &= A_n^{-1}b_n = [\Omega^{-1}\mu+n\Sigma^{-1}]^{-1}(b_o + b_1)\\
&=  [\Omega^{-1}\mu+n\Sigma^{-1}]^{-1}(\Omega^{-1}\mu+ \Sigma^{-1} n\bar{x} )
\end{align}
\vskip 1em
\begin{align}
\Sigma_n &= A_n^{-1} = [\Omega^{-1}\mu+n\Sigma^{-1}]^{-1}
\end{align}
}


\frame{
\frametitle{inverse Wishart  distribution}

Suppose $\Sigma \sim \text{inverseWishart}(\nu_o, S_o^{-1})$
where $\nu_o$ is a scalar and $S_o^{-1}$ is a matrix.
\vskip 1em

Then $$p(\Sigma) \propto
\det(\Sigma)^{-(\nu_o + p +1)/2} \times \exp\{
-\text{tr}(S_o\Sigma^{-1})/2
\}$$

\vskip 1em
For the full distribution, see Hoff, Chapter 7 (p. 110).
}

\frame{
\frametitle{inverse Wishart  distribution}

\begin{itemize}
\item The inverse Wishart distribution is the multivariate version of the Gamma distribution. 
\item The full hierarchy we're interested in is 
$$\bm{X} \mid \btheta, \Sigma \sim MVN(\btheta, \Sigma).$$ 
$$ \btheta \sim MVN(\mu, \Omega)$$
$$ \Sigma \sim \text{inverseWishart}(\nu_o, S_o^{-1}).$$
\end{itemize}
We first consider the conjugacy of the MVN and the inverse Wishart, i.e.
$$\bm{X} \mid \btheta, \Sigma \sim MVN(\btheta, \Sigma).$$ 
$$ \Sigma \sim \text{inverseWishart}(\nu_o, S_o^{-1}).$$
}

\frame{

What about $p(\Sigma \mid \bX,  \btheta) = p(\Sigma) \times p(\bX \mid \btheta, \Sigma).$
Let's first look at 
\begin{align}
&p(\bX \mid \btheta, \Sigma) \\
&\propto
\det(\Sigma)^{-n/2}\exp\{-
\sum_i (\bx_i - \btheta)^T\Sigma^{-1} (\bx_i - \btheta)/2
\}\\
&\propto 
\det(\Sigma)^{-n/2}\exp\{-
\text{tr}(S_\theta \Sigma^{-1}/2)
\}
\end{align}
Now we can calculate $p(\Sigma \mid \bX,  \btheta)$
}


\frame{
\begin{align}
&p(\Sigma \mid \bX,  \btheta) \\ & \quad= p(\Sigma) \times p(\bX \mid \btheta, \Sigma) \\
& \quad \propto 
\det(\Sigma)^{-(\nu_o + p +1)/2} \times \exp\{
-\text{tr}(S_o\Sigma^{-1})/2
\} \\
& \qquad \times
\det(\Sigma)^{-n/2}\exp\{-
\text{tr}(S_\theta \Sigma^{-1})/2\}\\
& \quad \propto 
\det(\Sigma)^{-(\nu_o + n + p +1)/2}
\exp\{-
\text{tr}((S_o +S_\theta) \Sigma^{-1})/2\}
\end{align}
This implies that 
$$\Sigma \mid \bX,  \btheta \sim \text{inverseWishart}(\nu_o + n, [S_o + S_\theta]^{-1} =: S_n)$$
}

\frame{
Suppose that we wish now to take 

$$\btheta \mid \bX, \Sigma \sim MVN(\mu_n, \Sigma_n)$$ (which we finished an example on earlier).
Now let $$\Sigma \mid \bX, \btheta \sim \text{inverseWishart}(\nu_n, S_n^{-1})$$
\vskip 1em There is no closed form expression for this posterior. Solution?

}

\frame{
Suppose the Gibbs sampler is at iteration $s.$
\begin{enumerate}
\item Sample $\btheta^{(s+1)}$ from it's full conditional:
\begin{enumerate}
\item[a)] Compute $\mu_n$ and $\Sigma_n$ from $\bX$ and $\Sigma^{(s+1)}$
\item[b)] Sample $\btheta^{(s+1)}\sim MVN(\mu_n, \Sigma_n)$
\end{enumerate}
\item Sample $\Sigma^{(s+1)}$ from its full conditional:
\begin{enumerate}
\item[a)] Compute $S_n$ from $\bX$ and $\Sigma^{(s+1)}$
\item[b)] Sample $\Sigma^{(s+1)} \sim \text{inverseWishart}(\nu_n, S_n^{-1})$
\end{enumerate}
\end{enumerate}


}

\frame{
\frametitle{Exercise}
We want to model the probability of recovery for patients admitted to the hospital in severe cardiac distress.
\vskip 1em
Suppose ``recovery'' means the patient survived long enough to be discharged from the hospital.
\vskip 1em
You have past data from a number of patients from 6 different hospitals.

For each patient $i$, you have various information 
\begin{itemize}
\item $x_i = (x_{i 1},\ldots,x_{i p})$ (e.g., gender, weight, age, blood pressure, etc.),
\item and a binary outcome $y_i$ (did the patient recover or not).
\end{itemize}
Design a Bayesian model for this problem.


}

%\frame{
%
%Many ways to approach this problem. 
%\vskip 1em
%One natural way is to use a hierarchical Probit regression model, as follows. 
%Let's divide up the patients according to hospital,
%and let $y_{hi}$ be the outcome for the $i$th patient in hospital $h$, and let $x_{hi} = (x_{hi1},\ldots,x_{hip})$ be the corresponding information.
%\begin{align*}
%    & Y_{hi}|\theta_h \,\sim\, \text{Bernoulli}(\Phi(\theta_h^T x_{hi})) \text{ independently} \\
%    & \theta_h|\mu,\Sigma \,\iid\, \text{MVN}(\mu,\Sigma) \\
%    & \mu \sim \N(\mu_0,\Sigma_0) \\
%    & \Sigma \sim \mathrm{InverseWishart}(S,\nu).
%\end{align*}
%
%}
%
%\frame{
%
%This would be appropriate if the effect of the predictors/covariates $x$ was expected to be different for the different hospitals.
%~\\~\\
%Another way would be to augment the covariate vector with indicator variables for the hospital, i.e., for $j = 1,\ldots,6$, define
%$x_{h,i,p+j} = \I(j = h)$, and use a Probit regression model with a Normal prior on $\theta = (\theta_1,\ldots,\theta_{p+6})$.
%This would be appropriate if the effect of the predictors/covariates was expected to be the same for the different hospitals,
%but the overall recovery rate varied among hospitals.
%
%
%}



\end{document}